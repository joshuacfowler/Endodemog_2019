\documentclass{article}

\usepackage[dvipsnames]{xcolor}
\newcommand{\tom}[1]{{\textit{\color{WildStrawberry}{[#1]}}}}

\title{Meeting Updates}
\author{Joshua Fowler}

%\usepackage{Sweave}
\begin{document}
%\input{MeetingUpdates-concordance}
%\SweaveOpts{concordance=TRUE}
  \maketitle
  Tom's comments are in \tom{WildStrawberry}.

  \section*{Feb 27, 2019}
\subsection*{Meeting Overview}
\textbf{We talked about the challenges from last week.}

\subsection*{Update on goals from last week}
\begin{itemize}
\item{I did not have results from the stochastic demography models due to memory issues. We decided to go back to single species models.}
\item{We went over the annotated bibliography for the herbarium project. I feel that I am at ~75 percent coverage of the literature. I think that there are interesting trends in what people have found looking at the relationship between endophytes and environmental factors. No one seems to have look at endophytes over time specifically. We talked about defining our research questions for this project and whether this satisfies the types of info we would want for the range limit question, and whether the temporal aspect is worth doing, or if there is some other broader question that would appropriate to look at with herbaria.}
\item{For the range limit project, we talked about the second draft of research questions. I think this was productive, and will continue to refine these.}
\end{itemize}

\subsection*{Goals for next week:}
\begin{itemize}
\item{For the stochastic demography: going back to single species models have results from the surv, growth, and flowering models and make a "beautiful figure".}
\item{For the Herbarium project: Continue to add to the annotated bibliography. Look specifically for surveys and patterns for the three species we have been talking about for the range limit project. Also, look for symbioses beyond endophytes/fungi.}
\item{For the Range limit project: Choose prospective sites for the transects in the experiment. Also, unpack the research questions more into separate questions.}
\end{itemize}

\section*{Feb 18, 2019}
\subsection*{Meeting Overview}
\textbf{This meeting was shorter, but we established some goals for next week's meeting. We talked about:}
\begin{itemize}
\item{Stochastic demography results from survival, growth and flowering models}
\item{Annotated bibliography for the herbarium project}
\item{2nd draft of questions for the range limit question and for the herbarium project}
\end{itemize}

\section*{Feb 13, 2019}
\subsection*{Meeting Overview}
\textbf{We talked about:}
\begin{itemize}
\item{A progress update on the stochastic demography project}
\item{For the herbarium project, we talked about the eventual size of this project, and how that will be driven by our questions and predictions for environmental change.}
\item{For the range limit project, we talked about the research question and about a general experimental design.}
\end{itemize}

\subsection*{Update on Goals from last meeting:}
\begin{itemize}
\item{\textbf{Stochastic demography:} The data is mergeable, and I have a script laid out for the growth model, but it does not account for the zero-inflated negative binomial. I need to look at vectorization for the year random effect still, and I have not had a true run of the models as written with loops.}
\item{\textbf{Herbarium project:} I have seed and some tissue samples from Mercer gardens. We did some seed squashes and saw endophytes which is really cool.}
\item{\textbf{Range limit project:} We talked about the questions and experimental design. I will continue to work on these. We didn't talk about this a ton, but thinking about Spring and Summer fieldwork, it looks like I will want to collect plants from across the range, and potentially along multiple transects. We also briefly talked about getting a sense of the seeds that we have on hand already.}
\item{\textbf{Committee meeting:}I have heard back now from everyone, and will go ahead with scheduling.}
\end{itemize}

\subsection*{Goals for next meeting:}
\begin{itemize}
\item{\textbf{Stochastic demography:} Incorporate model vectorization for the flowering and survival models and/or figure out if that is just not worth it. Add mixture model for the zero truncated distribution in the growth model.}
\item{\textbf{Herbarium project:} Perform a mini endophyte id experiment with the seeds from Mercer. I will primarily compare the seed squash and the agronostics. There is also the "pithe scraping" method, which I'm not sure that we have good enough samples for, but I will try to figure out what that method entails. 
Another goal is to be familiar with the endophyte range survey literature, and any similar herbaria project literature. This will be an ongoing process, but I would like to have something started for next week. This will also aid in refining the questions here.}
\item{\textbf{Range limit project:} Thinking about the experimental design in actual geographic space, locate possible sites for common garden plots. Related to the work with PRISM for the LTER project, I will spend some time working with climate maps, which I haven't had a ton of experience with. 
Bring draft 2 of the research question, revise it to be more specific.}
\item{\textbf{Committee meeting:} I will schedule this officially, and contact Rachael about a room. I will also make an outline presentation for myself, which will be the landing site for the various research questions that I am refining across the projects.}
\end{itemize}


\section*{Feb 7, 2019}
\subsection*{Progress Update}
\begin{itemize}
\item{\textbf{Stochastic demography:}
I have good progress on the flowering and seed data script. Overall, there is flowering tiller information for everything, barring a few NAs from data entry. There is information about the spikelets/inflorescence for a subset of the plants, but this is actually measured, and not estimated. It depends on the plant, but some have one inflorescence measured and some have up to 3 measured. There is seeds/spikelet information for even less plants than for spikelets. In the raw data sheets, there is often seed info, but it is basically just multiplied number of spikelets by some average that was calculated based on the seeds/spikelet data. So the way I am currently pulling in the data is to keep only the seed data that is actually measured. The rest are NA's, and I figure we can calculate that later. 

I have versions written up of the Survival and Flowering status models with the Species parameter. As we talked about, I have been trying to vectorize the  assignment statement for the prior for the year effect. I have gotten it to run with loops, but it is pretty slow, especially for the full dataset, so I think this could be worth doing, and talking to Bene about.
}

\item{\textbf{Herbarium:}
Currently, my plan is to follow the procedure that we outlined with Jessica. It sounds like they have good amounts of Lolium perenne, Agrostis hyemalis,  Elymus virginicus and Poa autumnalis. It's possible that they have other grasses, so having a list of possibilities would be useful, but I think it is mostly regional species, and that it is probably easier to get enough material from more common species. Hopefully I can get a bunch of seeds and we will find out if we can see endophytes. 

Before I go up there, I am going to put together a spreadsheet to keep track of what specimens we collect from which herbarium, and record the collection info. Having herbaria that are digitized will probably help us out but I think since we are handling each plant individually, it won't add much time. What time would work for us to chat about this before I go?

Longer term, I think it makes sense to put together candidate herbaria that I want to visit based on their collections, or locations. There will be a fair amount of work to be able to contact them ahead of time and schedule times to visit, and doing this sooner rather than later will be important if I want to do this throughout this semester/summer. Getting a lot of the collecting done this summer is my plan, but we can also continue to visit new herbaria over time because the specimens will still be there.  

I have a general idea that as many plants as possible is good, but maybe we can talk about how many we actually are aiming for. Based on the searching I did through the UT Austin database, It seems like it is not totally uncommon for there to be multiple collections over time within a county. I think this is partially related to the question of how good of an estimate of endophyte prevalence do we have in a location based on one plant. Sometimes there are multiple collections from the same county potentially within the same year or a few years, which would be useful to have, but I think would probably be something that I will need to justify to herbarium directors.}

\item{\textbf{Range Limit project:}
I've identified research questions and I'm attaching a picture of a sketch of the experimental design. I want to make sure that this is not just asking the question, what is controlling the range limit for these grasses.
      Questions: 
                Generally, how do mutualistic biotic interactions contribute to species range limits?
                Given that they are often context dependent, where do mutualists have the biggest influence?

 Within this picture, I am including 5 sites, but having more is good. By "paired plots" I mean that for each env - competition treatment, we would have an E+ plot and an E- plot that are next to each other, and we would have more than one of these pairs. We can't plant them in the same plot because we want to track the recruits and know the endophyte status. Within the plot, I would have multiple plants that are a random mix of plants collected from across the range. This is opposed to having a plot for each population that we collect from, and having multiple populations planted at each location would minimize the effect of local adaptation. This is also what it would look like for 1 species. I think it would make sense to have separate plots with this same set up for other species. I don't know how difficult it is to id different species as recruits, but I think that we would also have to contend with them competing with each other in that situation.  I also think it would be cool to have multiple transects, one south and one north, but I'm not sure how logistically feasible that is.}

\item{\textbf{Lab Greenhouse project:}
The green house portion of Michelle Afkhami's range limit paper has a comparison of E+, E-, and E+-, but the treatment is a fungicide. So if I'm reading it correctly, they have natural E+ and E minus, and E+- and E- fungicide. I think it's also not uncommon for people to do at least some sort of comparison between the control and the treatment, although I don't think it is always included in the full experiment, or as a main aim, just that there is often a line to justify that their treatment control is valid.}
\end{itemize}

\section*{Jan 30, 2019}
\subsection*{Meeting Overview}
\textbf{We talked about:}
\begin{itemize}
\item{Committee meeting scheduling}
\item{Agrinostics data analysis}
\item{Stochastic demography project}
\item{Herbarium project}
\item{Range limit project}
\item{Spring fieldwork schedule}
\end{itemize}

\subsection*{Updates on goals from last meeting:}
\begin{itemize}
\item{I am working on an email to send out to contact herbaria. I had been thinking about UT Austin, but based on our meeting, I have contacted Mercer Gardens herbarium which seems like a promising local herbarium, and can also contact SHSU. I am planning to try to get seeds in this first bout of collection for testing the different methods of endophyte id.\tom{I saw in your email that you also intend to get tiller tissue, so it would be good to have a plan for what you will do with it.}}
\item{I have mostly finished the data script, but I need to work on making sure that it is actually doing all the right things, and that I can merge it with the endo-demog-long file from Jenn.}
\item{I have built up the survival model to include a species effect, but I need to run it still, and will try to rerun the survival and flowering models for next week.}
\item{Volker, Jenn, and Lydia have agreed to be a part of the committee. Nothing is scheduled yet.}
\end{itemize}

\subsection*{Goals for next meeting:}
\begin{itemize}
\item{\tom{I think it would be preferable to organize goals by project. Each project really has its own list.}}
\item{I will send out a committee scheduling poll to Lydia, Volker, Jenn and Tom, probably on Friday.}
\item{I will send out the email to Mercer Gardens herbarium hopefully hopefully on Friday. It seems like it would maybe be possible to go out there this coming week, or next weekend. One thing I would like to do is learn more about handling and taking tissue from herbaria specimens.}
\item{For the stochastic demography project, my main goal for next week is to merge the flowering data with the main data. I will run the survival and flowering models for next week. Potentially, I will try it with a plot random effect as well.}
\item{For the range limit project, I will write out the questions and sketch out an experimental design for next week. Hopefully we can evaluate what I come up with, and then that will inform Spring and Summer fieldwork scheduling.}
\end{itemize}

\subsection*{Semester-level goals:}
\begin{itemize}
\item{Obtain herbarium seeds to test by Feb. 13th-ish.\tom{This is not really a semester-level goal}}
\item{Build up list of herbaria to contact and of specific specimens to look for, so that after our test seeds, we can keep moving forward, and maybe visit as many local herbaria during the semester as possible. Then this Summer, visit some that are farther afield.\tom{I think that the questions motivating the herbarium work are not yet fully formed, and this should precede any herbarium trips.}}
\item{Build growth model by Feb. 13.}
\item{Build seed production model by Feb. 27 \tom{I am not really sure what this means.}}
\item{Build flw tiller production model by end of March, depending on progress with growth and seed production models}
\item{Committee meeting week of March 3rd, 10th, or 17th.}
\item{Have all parts of the stochastic population model by end of semester (week of May 1st), at least working. \tom{My suggestion is that they should be final by then.}}
\item{Collect seeds and start growing plants for range limit experiment. I will work on the experimental design to think about more specific plans/dates for this, but collecting April-ish with a general plan of growing over the Summer and planting out in the late Fall.}
\end{itemize}



\section*{Jan 23, 2019}
\subsection*{Meeting Overview}
We talked about geographic patterns in rates of climate change. Overall, the SE US has not seen the largest historic changes in temperature, and Indiana and Texas have both seen climate change in slightly different ways. There is a general pattern of more precipitation across Texas going north. This led us to general agreement to focus on the Texas region/Eastern species with western range edges because it also lines up nicely with the precipitation gradient that exists here. We also briefly talked about the Southwest or other areas that have experienced climate change around the US. Following off of this, we talked about getting some seeds to test if we can find endophytes. We also talked about adding a competition treatment to the range limit experiment, which I think could be valuable based on my current beliefs about the sorts of things that are contributing to the Texas ranges, but needs to be more fleshed out. We also talked Thursday about the possibility that we can use only herbarium specimens and not resample sites which could make sense to have the two projects less connected to each other.\tom{This is a good recap, but it mixes up the herbarium project and the range limits project, which are obviously related but different. Going forward I think it would be good to provide a recap on each of your projects/activities (these two things, plus the stochastic demography project, plus the endophyte elimination experiment).}

\subsection*{Goals for next meeting:}
\begin{itemize}
\item{I am in contact with Jessica Budke, and I will contact some regional herbaria. I think it would be great to have a concrete plan towards getting seeds set up by next week's meeting, and then potentially getting seeds by the two weeks following. This will include:
\begin{itemize}
\item{Choosing species}
\item{Choosing herbarium records}
\item{Traveling to get the samples. There is a possibility of "loans" but that is something I will find out about.}
\end{itemize}}
\item{I will keep working on the data script. I would like to have a mostly finalized version by next week. I have been in contact with Jenn which is useful.}
\item{I built up the survival model to run with a species effect, but haven't played around with it too much to see how it fits. I want to have a run of this model by next week and potentially do the same for the flowering model.}
\item{We also talked about my committee meeting. I will ask Volker and Lydia so that we can start planning that.}
\end{itemize}

\tom{Here is what I see as the agenda for our Tuesday meeting:
\begin{itemize}
\item{Update from Josh about committee meeting}
\item{Update on agrinostics screening and plan for data analysis.}
\item{Stochastic demography analyses: udpates from Josh related to goals above}
\item{Herbarium project: updates from Josh related to goals above}
\item{Range limits experiment...not sure what the goals here were.}
\item{Discussion of spring field work.}
\end{itemize}
}

\section*{Jan 16, 2019}
\subsection*{Meeting Overview}
We talked about working on two general areas for this semester. I have a goal of finishing the IPM for the Endo variability project by the end of the semester. We also talked about field work for the range limit questions.

\subsection*{Goals for next meeting:}
\begin{itemize}
\item{Refine Herbarium questions and hypotheses. Come up with mock plan for how to test these. I will send Jessica Budke an email before our next meeting.}
\item{Look into rates of climate change thinking about ranges where herbarium studies would be interesting.}
\item{Finalize data script by Jan. 30th. I will send Jenn an email before our meeting next week for some clarifications.}
\item{Write up the Survival model so that it can be run with a species effect by our next meeting.}
\item{Longer term: I will start coding up the stan models for Growth, seed production, and recruitment. I will plan to start with Growth and go from there. Hopefully starting working on that after Jan. 30th.}
\end{itemize}



\section*{Dec 15, 2018}
\subsection*{Winter Break Time Line}
Overall, we decided to work on the three vital rate models for survival, growth, and flowering tillers. We also need to do model diagnostics and work on figures for the EvoDemo meeting on Jan 9th. The following is a general schedule where I am setting deadlines, but will plan that much of this can be worked on concurrently.

\subsection*{Schedule:}
\begin{itemize}
\item{Dec 21: I want to have a functioning version of the survival model at this point. We talked during our last meeting about how to add in the endo effect to be able to analyze the variance. I will work on incorporating this as a term within the year random effect. \tom{I think there are two ways of doing this. The way that we wrote out would have E+ and E- plants having different temporal distributions on the intercept values. A second way would be to put a temporal distribution on an endophyte effect parameter. These would be equivalent models that use a different parameterization, essentially a `mean parameterization' versus an `effect parameterization'. For now, proceed with whatever makes the most sense to you and we can discuss.}}
\item{Dec 24: Finish the Flower model. This should be pretty similar to the Survival model. I will also start looking at model diagnostics and think about visualizing the data once the models are running.\tom{Yes, flowering will be easy once you do survival.}} 
\item{Dec 28: Finish the Growth model. This is different from the other two in that it will require a negative binomial distribution.\tom{Yes, there are several ways to parameterize a NB. The easiest (and hackiest) way to do this would be to use a Poisson model but add an individual-level random effect. This basically adds overdispersion, which allows the Poisson to approximate the NB. Also remember that this will need to be zero-truncated. This is easy to do in BUGS, not sure about Stan (google it).}}
\item{Dec 29 - Jan 4: Run the models, and save the outputs. During this, time, I will also finish up model diagnostics and figures.}
\item{Jan 6th: Back at Rice\tom{Let's schedule a meeting first thing when you are back, so you can show me what should basically be all the raw materials for your poster.}}
\item{Jan7-8th: Print poster}
\item{Jan 9th: Travel to Miami}
\end{itemize}

\section*{Nov 15, 2018}
\tom{This all looks good to me. Coding up the interaction models in Stan may be the most challenging part so it is worth getting this sorted out soon. As you know, I have not done this myself, but I am certain we can figure it out together.}

\subsection*{Meeting Overview}

We talked about my progress with the models and the data script. I didn't make as much progress with the interactive terms because Stan has been giving me an error I think due to how I am loading in the data and calling it within the model. I will try to have that figured out for next week. I want to find some resources about vectorizing the code with random effects because I think that is part of the issue, as well as vectorizing just not being totally clear to me yet. We also talked about learning more about choosing distributions. 

\subsection*{Goals for Next Meeting}
\textbf{Goals include:}
\begin{itemize}
\item{I want to make progress with the data script. I can have all the species (but probably not the seed data frame done yet) for next Thursday, and then we could start using that data for the survival model. I can hopefully have both the regular and the seed data done by the following week, when we will have our next Thursday meeting. I will have looked more into that this coming week, and so maybe we can talk about how we are doing the seed estimate.} 
\item{For 2 weeks from now, I want to have my error message figured out so that I can run the models without continuing to have these problems. Then I think it should be relatively easy to add in fixed parameters, especially once I have the full dataframe.}
\end{itemize}

\subsection*{Goals continuing from last Meeting}
\textbf{Goals include:}
\begin{itemize}
\item{For Tues., Nov. 20, have 1-2 paragraph intro for the proposal.}
\item{For Tues., Nov. 20, look at maps for the endodemog focal species.}
\item{For Thurs., Nov. 22, Thanksgiving}
\item{For Tues., Nov. 27, have a rough draft of the core course proposal for Tom to review.}
\item{For Thurs., Nov. 29, rough draft of the proposal is due for core course.}
\item{For Thurs., Nov. 29, have a rough draft of a Science Day presentation.}
\item{For Thurs., Nov. 29, finish up the data script, both the main sheet and the seed production sheet.}
\item{For Mon., Dec. 3, Science Day!}
\item{For Thurs, Dec. 6, the core course proposal is due.}
\item{For Mon., Dec. 7, the NDSEG is due.}
\item{Longer term, we have the EvoDemo Meeting Jan. 10th-12th.}
\end{itemize}

\section*{Nov. 13, 2018}
\subsection*{Meeting Overview}

We talked about my "Top 8" papers list. Basically, there is a common thread that we are developing between the papers that includes using a comprehensive demographic approach to study range edges and how biotic interactions influence these range edges,  while thinking about how elevational vs latitudinal gradients compare, and the herbarium idea. 

\subsection*{Goals for Next Meeting and Beyond}
\textbf{Goals include:}
\begin{itemize}
\item{For next Tuesday, come up with a first paragraph for the end of semester proposal that will be the quick literature review of the problems on which we are focusing.}
\item{For next Tuesday, look up distribution maps for the species of grasses from the Endodemog data set}
\end{itemize}

\subsection*{Goals continuing from last Meeting}
\textbf{Goals include:}
\begin{itemize}
\item{I still need to send out the information about the NDSEG. I will do that on Wednesday.}
\item{See the previous meeting for goals for the Endodemog model and data script.}
\end{itemize}


\section*{Nov. 8, 2018}
\subsection*{Meeting Overview}

This section includes the meeting information for both this and the preceding Tuesday meeting. We talked about my gathering bibliography and planned to have a "Top 10" List for the following Tuesday. During the Thursday meeting, we talked about the data script and then discussed full linear predictor for the endodemog model.

\subsection*{Goals for Next Meeting and Beyond}
\textbf{Goals include:}
\begin{itemize}
\item{Add top 10 papers to the annotated bibliography. Those that encapsulate the ideas that I am interested in and are potentially going to be foundational.}
\item{Continue to work on the data cleaning script. For next Thursday, I would like to pull out the seedling data for 1 species, and get started on that, as well as continue to add species to the main dataframe.}
\item{We will work on the implementing the full model for Survival. I'm not sure if I will be able to have this done, but I will try to add in one of the interacting effects for next week.}
\end{itemize}

\subsection*{Goals continuing from last Meeting}
\textbf{Goals include:}
\begin{itemize}
\item{I need to send out information to my reference writers for the NDSEG still. I am hoping to work all of the various end of the semester projects kind of at the same time, which is a little bit overwhelming, but I think that as I make progress on the core course proposal and write up the NDSEG, it will all come together.}
\end{itemize}

\section*{Oct. 30, 2018}

\subsection*{Meeting Overview}

We discussed my progress in defining a project. We talked about how we have done some backwards type exploration and we want to push forwards to meet in the middle with the literature review. We basically talked about lining these up to fit the timeline of the proposal for core course, my application for the NDSEG, Science day, for the EvoDemo conference, and eventually for this Summer.
 
\subsection*{Goals for Next Meeting and Beyond}
\textbf{Goals include:} 
\begin{itemize}
\item{Tom adds comments to the coreview by Thursday, update and send to New Phytologist by Friday.}
\item{Have the annotated bibliography filled out at least as far as having the citations for literature related to fungal endophyte surveys, other related symbionts like wolbachia, and range limits with biotic interactions}
\item{I will look this weekend at the application for the NSDEG fellowship and provide information for the letters of reference.}
\end{itemize}

\subsection*{Goals continuing from last Meeting}
\textbf{Goals include:} 
\begin{itemize}
\item{For the data cleanup script, I need to read about relational databases, as I am at the point where I have gotten pretty far (melt and cast have been very helpful) with the main survival/growth sheet for Poal, but will have to set up the data for the seed production estimates.}
\item{For continuing to develop my research ideas, I am doing the annotated bibliography and exploring other ideas. Both of these are relevant to the Core Course final proposal, as well as Science Day presentations, and partially the EvoDemo conference. The core course final proposal is my current deadline for having a mostly finished annotated bibliography of the main ideas that I want to research.}
\end{itemize}


\section*{Oct. 16, 2018}

\subsection*{Meeting Overview}

We talked briefly about the Hertz fellowship and then we discussed geographic mosaic theory.Tom said he would send a couple of papers (I don't remember who by, but about range limits and gene flow). Overall, we will continue to think about these other topics, especially how they may be complementary to thinking about range limits, but we are going to focus on making progress on the modeling and on developing an annotated bibliography for the range limits questions. We also set a timeline for the co-review.

\subsection*{Goals for Next Meeting and Beyond}
\textbf{Goals include:} 
\begin{itemize}
\item{Read paper for co-review by Tuesday. I don't remember exactly what we laid out for the rest, but basically comments for the week following, then I will write up the review and give it to you to adjust. Can you add the timeline we talked about?}
\item{Exploring other topics is partially on the backburner. See my continuing goals.}
\end{itemize}

\subsection*{Goals continuing from last Meeting}
\textbf{Goals include:} 
\begin{itemize}
\item{For data cleanup script, I need to go through the raw data files, and loading them into Rstudio. I would like to have this done by next week. The next step is that I need to pull out the important columns and combine sheets. This is also the stage where the relational database will come in because we have to include the seed count data.}
\item{For the model, I will include post. pred. checks by next Thursday. I think I will probably not start the Growth model until after next week but a good place to start is visualizing the data and writing the most simple Stan Model (using the Poisson distribution). I will put a hopeful goal of this for Thursday, Nov. 1st.}
\item{Edit Hertz application essays, and finish the personal informationand experience side of the application. Due Dates: \begin{itemize}
\item{Feedback on my essays = Oct 19th} 
\item{Josh's Application = Oct 24th} 
\item{Reference Letters = Oct 26th}
\end{itemize}}
\item{For continuing to develop my research ideas, I am doing the annotated bibliography and exploring other ideas. Both of these are relevant to the Core Course final proposal, as well as Science Day presentations, and partially the EvoDemo conference. The core course final proposal is my current deadline for having a mostly finished annotated bibliography of the main ideas that I want to research}
\item{For the annotated bibliography, identify the key papers that I am working from, like the Angert papers, for next week. I think it would be good to develop a list of my key words as well.}
\end{itemize}

\section*{Oct. 11, 2018}

\subsection*{Meeting Overview}

We discussed our survival model with year effects, and briefly looked at the model output. We talked about starting to work with the raw data, and potentially starting to look at working on the growth model. We also talked about the EvoDemoSoc meeting, which would be really cool to attend.

\subsection*{Goals for Next Meeting and Beyond}
\textbf{Goals include:} 
\begin{itemize}
\item{Build script to clean up data}
\item{Get year variance from model, and visualize posterior predictive checks}
\item{Figure out my computer's Stan error}
\item{Start to work on Growth (start with Poisson distribution) \tom{When you are ready for it, I can share an example of negative binomial growth modeling in Stan.}}
\end{itemize}

\subsection*{Goals continuing from last Meeting}
\textbf{Goals include:} 
\begin{itemize}
\item{Work on Hertz application}
\item{Work on annotated bibliography and continue to look at systems and logistics \tom{See comment on same point below}}
\item{Brainstorm other research ideas: Reading about Geographic mosaic theory of coevolution for Tuesday, Oct. 16th}
\end{itemize}


\section*{Oct. 3, 2018}
\tom{I made the notes below before pulling some of the more recent meeting notes - sorry. Some of these are not as relevant given your points above.}
\subsection*{Meeting Overview}

We discussed the Hertz Fellowship, predictions and next steps for the transplant experiment, as well as other possible research directions to explore.

\subsection*{Goals for Next Meeting and Beyond \tom{Not clear which of goals below are near-term action items vs longer-term stuff.}}
\textbf{Goals include:} 
\begin{itemize}
\item{Work on Hertz application} \tom{What does `work' mean? What do you need to do?}
\item{Start on annotated bibliography and continue to look at systems and logistics \tom{How will you do this? What steps will you take?}}
\item{Brainstorm other research ideas}
\item{Work on Endodemog model with year effects and visualizing the model for Stan \tom{This is a good, concrete goal. We talked about posterior predictive checks and I think this should be included in your modeling work.}}
\item{Work on Endodemog data cleaning script \tom{Again, what are the steps involved here and what is your timeline for completing them?}} 
\item{\tom{Obviously, most of my suggestions here are about making your work - and your ability to hold yourself accountable - more explicit with dates and specific steps.}}
\end{itemize}
\end{document}
